%%%%%%%%%%%%%%%%%%%%%%%%%%%%%%%%%%%%%%%%%
% Medium Length Professional CV
% LaTeX Template
% Version 2.0 (8/5/13)
%
% This template has been downloaded from:
% http://www.LaTeXTemplates.com
%
% Original author:
% Rishi Shah 
%
% Important note:
% This template requires the resume.cls file to be in the same directory as the
% .tex file. The resume.cls file provides the resume style used for structuring the
% document.
%
%%%%%%%%%%%%%%%%%%%%%%%%%%%%%%%%%%%%%%%%%

%-------------------------------------------------------------------
%	PACKAGES AND OTHER DOCUMENT CONFIGURATIONS
%-------------------------------------------------------------------

\documentclass{resume} % Use the custom resume.cls style

\usepackage[left=0.75in,top=0.6in,right=0.75in,bottom=0.6in]{geometry} % Document margins
\usepackage[colorlinks = true,
            linkcolor = blue,
            urlcolor  = blue,
            citecolor = blue,
            anchorcolor = blue]{hyperref}
\newcommand{\tab}[1]{\hspace{.2667\textwidth}\rlap{#1}}
\newcommand{\itab}[1]{\hspace{0em}\rlap{#1}}
% Use multiple bibs
\usepackage[resetlabels]{multibib}
\newcites{peer}{Peer Reviewed Pubs}
\newcites{unref}{Unrefereed Pubs}
% Document Information
\name{Marion S Alberty}
\address{201 Forrestal Road, Princeton, NJ 08542} % primary address
\address{(+1) 330-259-5864 \\ \href{mailto:malberty@princeton.edu}{malberty@princeton.edu} \\ \href{mailto:marion.s.alberty@gmail.com}{marion.s.alberty@gmail.com}}
\begin{document}

%-------------------------------------------------------------------
%	EDUCATION SECTION
%-------------------------------------------------------------------
\begin{rSection}{Education}
{\bf University of California San Diego (UCSD)} \hfill {\em Aug 2012 - Dec 2018}\\
{\bf Scripps Institution of Oceanography (SIO), La Jolla, CA} \\ Ph.D. in Oceanography \\
\\
{\bf Cornell University, Ithaca, NY} \hfill {\em Aug 2008 - May 2012} \\
B.Sc. in Civil \& Environmental Engineering
\end{rSection}

%-------------------------------------------------------------------
%	EXPERIENCE SECTION
%-------------------------------------------------------------------
\begin{rSection}{Appointments}
{\bf Postdoctoral Research Associate} \hfill {\em Feb 2019 - present}\\
Geophysical Fluid Dynamics Laboratory, Atmospheric and Oceanic Sciences, Princeton University, Princeton, NJ \\
{\bf Graduate Ph.D. Researcher} \hfill {\em Aug 2012 - Jan 2019}\\
Sprintall Group and Multiscale Ocean Dynamics Lab, SIO, La Jolla, CA \\
{\bf Graduate Student Researcher} \hfill {\em Jul-Aug 2012}\\
Send Lab, SIO, La Jolla, CA \\
{\bf Undergraduate Student Researcher} \hfill {\em Aug 2011 - May 2012}\\
Cowen Lab, Cornell University, Ithaca, NY
\end{rSection}

%-------------------------------------------------------------------
%    Publications
%-------------------------------------------------------------------
\begin{rSection}{Peer-Reviewed Publications}
\href{https://scholar.google.com/citations?user=UGkNcW8AAAAJ&hl=en&oi=sra}{Link to google scholar page}
\begingroup
\renewcommand{\section}[2]{}
\nocitepeer{*}
\bibliographystylepeer{plainyr-rev}
\bibliographypeer{AlbertyPeerBib}
\endgroup
\end{rSection}

\begin{rSection}{Unrefereed Publications}
\begingroup
\renewcommand{\section}[2]{}
\nociteunref{*}
\bibliographystyleunref{plainyr-rev}
\bibliographyunref{AlbertyUnrefBib}
\endgroup
\end{rSection}

%-------------------------------------------------------------------
%	Conferences and Meetings
%-------------------------------------------------------------------
\begin{rSection}{Conferences and Meetings} 
\item Alberty, M., S. Legg, R. Hallberg, J. MacKinnon, J. Sprintall, M. Alford, J. Mickett, \& E. Fine, February 2020: The Impact of Submesoscale Dynamics on Arctic Freshwater Fronts (poster), Ocean Sciences Meeting, San Diego, CA.
\item Alberty, M., J. MacKinnon, \& J. Sprintall, June 2018: Observations of Submesoscale Mixed Layer Processes in the Arctic (poster), Gordon Research Conference on Ocean Mixing, Andover, NH.
\item Alberty, M., J. Sprintall, S. Cravatte, A. Ganachaud, C. Germineaud, J. MacKinnon, \& G. Eldin, February 2018: Moored Observations of Equatorward Transport in the Solomon Sea, 2018 Ocean Sciecenes Meeting, Portland, OR.
\item Alberty, M., J. Sprintall, J. MacKinnon, A. Ganachaud, S. Cravatte, \& C. Germineaud, May 2017: \href{http://www.legos.obs-mip.fr/members/cravatte/alberty-talk}{Spatial Patterns of Mixing in the Solomon Sea} (plenary talk), Solomon Sea Oceanography Workshop, Toulouse, France. 
\item Alberty, M., J. MacKinnon, \& J. Sprintall, May 2016: Observation of submesoscale features in the Canada Basin (poster), 48th Liege Colloquium on Submesoscale Processes, Liege, Belgium.
\item Alberty, M., J. Sprintall, J. MacKinnon, A. Ganachaud, S. Cravatte, \& C. Germineaud, February 2016: Preliminary Observation of AAIW in the Solomon Sea (poster), Ocean Sciences Meeting, New Orleans, LA. 
\item Alberty, M., J. Sprintall \& J. MacKinnon, June 2015: Spatial and Temporal Variability of Mixing in the Solomon Sea, International Union of Geodesy and Geophysics; International Association for the Physical Science of the Ocean, Prague, Czech Republic.
\item Alberty, M., J. Sprintall \& J. MacKinnon, April 2015: Water Mass Modification through Mixing in the Solomon Sea (poster), European Geosciences Union, Vienna, Austria.
\item Alberty, M., J. Sprintall \& J. MacKinnon, November 2014: Spatial Patterns of Mixing in the Solomon Sea, International Meeting of Students in Physical Oceanography, Ensenada, Mexico.
\item Alberty, M., J. Sprintall \& J. MacKinnon, June 2014: Observing Internal Waves in the Solomon Sea (poster), Nonlinear Effects in Internal Waves, Ithaca, NY.
\item Alberty, M., \& U. Send, August 2012: Analysis of deltapCO2 Time Series in the Central California Ocean Margin (Poster), International Meeting of Students in Physical Oceanography, La Jolla, CA. 
\end{rSection}

%-------------------------------------------------------------------
%	AWARDS SECTION
%-------------------------------------------------------------------
\begin{rSection}{Honors and Awards}
Wyer Family Fellowship \hfill {\em Sep 2017 - Dec 2018} \\
NASA Earth and Space Fellowship \hfill {\em Sep 2016 - Dec 2018} \\
Dr. John Roads Endowed Fellowship \hfill {\em Sep 2015 - Aug 2016} \\
STEM Chateaubriand Fellow at Laboratoire d’Etudes en \hfill {\em Jan - Jul 2015} \\
G\'eophysique et Oc\'eanographie Spatiales (LEGOS), Toulouse, France \\
UC Regents First-Year Fellowship, George Mitchell \hfill {\em Sep 2012 - Aug 2013} \\
Fellowship \& Arete Associates - Jacobs Fellowship \\
Engineering Learning Initiatives Undergraduate Research Award \hfill {\em Aug 2011 - May 2012}
\end{rSection}

%-------------------------------------------------------------------
%	TEACHING SECTION
%-------------------------------------------------------------------
\begin{rSection}{Teaching}
\item {\bf Guest Lecturer} GEO425 Introduction to Ocean Physics for Climate, Princeton University, {\em Fall 2019} (Lecture recording available on request)
\item {\bf Co-Instructor} Prison Teaching Initiative MATH020 Elementary Algebra, Edna Mahan MAX, Clinton, NJ, {\em Fall 2019}
\end{rSection}

%-------------------------------------------------------------------
%	MENTORING SECTION
%-------------------------------------------------------------------
\begin{rSection}{Mentoring}
% Undergrad mentees
\begin{rSubsectionW}{Undergraduate Students}
\\Akira Disandro, Cooperative Institute for Modeling the Earth System (CIMES) Research Internship Program at Princeton University (remote), \href{https://github.com/disandroa/GFDL_Notebooks}{Validating Tropical Pacific Circulation in GFDL Ocean Models}, Summer 2020
\end{rSubsectionW}
\\
% Mentors
\begin{rSubsectionW}{Graduate and Postdoctoral Advisors}
\\Sonya Legg, Princeton University, Postdoctoral advisor\\
Janet Sprintall \& Jennifer MacKinnon, Ph.D. advisors
\end{rSubsectionW}
\end{rSection}

%-------------------------------------------------------------------
%	SERVICE SECTION
%-------------------------------------------------------------------
\begin{rSection}{Service}
% Conference related
\begin{rSubsection}{Conference Chair}
Gordon Research Seminar on Ocean Mixing, Planned June 2022, South Hadley, MA
\end{rSubsection}
% Reviewer
\begin{rSubsectionW}{Reviewer}
\\Geophysical Research Letters, Progress in Oceanography
\end{rSubsectionW}
\\
% Committees
\begin{rSubsection}{Committee Member}
\item {\em 2020}: GFDL Diversity, Equity, and Inclusivity
\item {\em 2016}: SIO faculty search in polar sciences, student committee
\end{rSubsection}
% Outreach
\begin{rSubsection}{Outreach}
\item {\em 2020}: Organized the remote 2020 CIMES Summer Internship Program's mini lecture series, Pangeo Tutorial, and community slack channel
\item {\em 2019}: Young Women's Conference in Science, Technology, Engineering \& Mathematics, Princeton Plasma Physics Laboratory
\item {\em 2014-2018}: Two Scientists Walk into a Bar, Reuben H. Fleet Science Center
\item {\em 2015-2018}: Rosa Parks Tutoring Program
\item {\em 2018}: Exploring Ocean STEM Careers Night, Birch Aquarium at Scripps Institution of Oceanography
\item {\em 2016-2017}: Beach Science program with Birch Aquarium
\end{rSubsection}
% Membership
\begin{rSubsectionW}{Membership}
\\The Oceanography Society, MPOWIR
\end{rSubsectionW}
\end{rSection}

%-------------------------------------------------------------------
%	Field Experience
%-------------------------------------------------------------------
\begin{rSection}{Field Experience} 
\item {\bf 2016}: Small boat and kayak-based work observing frontal features, eddy formation and lake dynamics, Palau
\item {\bf 2015}: {\em ArcticMix}, small scale process-study, {\em R/V Sikuliaq}; Nome to Nome. PIs: Jennifer MacKinnon, Matthew Alford, and John Mickett
\item {\bf 2014}: {\em La Jolla Internal Tide experiment} (student-run cruise), {\em R/V Sproul}; San Diego to San Diego. Chief Scientist: Madeline Hamman
\item {\bf 2014}: {\em MoorSPICE}, mooring recovery and reployment cruise, {\em R/V Thomas G. Thompson}; Noumea to Noumea. PIs: Janet Sprintall and Sophie Cravatte
\item {\bf 2012}: {\em California Current Ecosystem} redeployment cruise, {\em R/V Ocean Starr}; Port Hueneme to Port Hueneme. PI: Uwe Send
\item {\bf 2011-2012}: Small boat work on Cayuga Lake, Ithaca, NY. Water quality monitoring, deployment and recovery of mooring temperature network, ADCP transects, met station maintenance. PIs: Tod Cowan and Seth Schweitzer
\end{rSection}

%-------------------------------------------------------------------
%	SKILLS
%-------------------------------------------------------------------
\begin{rSection}{Skills}
\begin{tabular}{ @{} >{\bfseries}l @{\hspace{6ex}} l }
Programming \& Modeling & \href{https://github.com/marionalberty}{Python, MATLAB, git, MITgcm}  \\
In-Situ Data Processing & Shipboard and wire walker CTD, SADCP/LADCP \\
 & moored T/TP/CTD/ADCP \\
\end{tabular}
\end{rSection}

% Date of last update
\vspace*{\fill}
\hfill \it{This CV was last updated \today}
\end{document}